\begin{center}
	\hrule
	\vspace{.4cm}
	\Large\scshape\textbf{AQM 831 --- Advanced Quantum Mechanics-1}
\end{center}
{\textbf{Name:}\ \textsc{Akiful Islam Zawad} \hspace{\hfill} \textbf{Due Date:} August 26 2024\\[5pt]
{ \textbf{Track Follower:}} \ 2 \hspace{\hfill} \textbf{Week Number:} 4 \\
	\hrule}
 %----------------------------
\paragraph*{Problem Set 4} %\hfill \newline
\\
Consider the matrix
\begin{align*}
    A = \begin{bmatrix}
    2 & 1 & 0 \\
    0 & 3 & 0 \\
    1 & 0 & 4
    \end{bmatrix}
\end{align*}
Find the characteristic equation, eigenvalues, and the corresponding orthonormal eigenvectors.
\bigskip\bigskip\hline\hline\bigskip
\section*{Solution}
The characteristic equation of a matrix $A$ is obtained by solving the determinant of the matrix $A - \lambda I$ set equal to zero, where $\lambda$ represents the eigenvalues and $I$ is the identity matrix.

First, we write the matrix $A - \lambda I$, where $I$ is the $3 \times 3$ identity matrix:

\begin{align}
    A - \lambda I &= \begin{bmatrix}
    2 & 1 & 0 \\
    0 & 3 & 0 \\
    1 & 0 & 4
    \end{bmatrix} - \lambda \begin{bmatrix}
    1 & 0 & 0 \\
    0 & 1 & 0 \\
    0 & 0 & 1
    \end{bmatrix}\notag\\
    A - \lambda I &= \begin{bmatrix}
    2 - \lambda & 1 & 0 \\
    0 & 3 - \lambda & 0 \\
    1 & 0 & 4 - \lambda
    \end{bmatrix}
\end{align}
Compute the Determinant of $A - \lambda I$.
\begin{align}
    \det(A - \lambda I) = \det\left(\begin{bmatrix}
    2 - \lambda & 1 & 0 \\
    0 & 3 - \lambda & 0 \\
    1 & 0 & 4 - \lambda
    \end{bmatrix}\right) = 0
\end{align}
To compute this determinant, expand along the first row:
\begin{align}
    \det(A - \lambda I) = (2 - \lambda) \begin{vmatrix}
    3 - \lambda & 0 \\
    0 & 4 - \lambda
    \end{vmatrix} - 1 \begin{vmatrix}
    0 & 0 \\
    1 & 4 - \lambda
    \end{vmatrix} + 0 \begin{vmatrix}
    0 & 3 - \lambda \\
    1 & 0
    \end{vmatrix}\label{eq:det-first-step}
\end{align}
First, calculate the $2 \times 2$ determinant for the first term:
\begin{align}
    \begin{vmatrix}
    3 - \lambda & 0 \\
    0 & 4 - \lambda
    \end{vmatrix} = (3 - \lambda)(4 - \lambda) = 12 - 7\lambda + \lambda^2\notag
\end{align}
The second and third terms are simplified as follows:
\begin{align}
    \begin{vmatrix}
    0 & 0 \\
    1 & 4 - \lambda
    \end{vmatrix} = 0 \quad \text{and} \quad \begin{vmatrix}
    0 & 3 - \lambda \\
    1 & 0
    \end{vmatrix} = -(3 - \lambda)\notag
\end{align}
Substitute these into (\ref{eq:det-first-step}):
\begin{align}
    \det(A - \lambda I) = (2 - \lambda)(12 - 7\lambda + \lambda^2) - 0 + 0
\end{align}
Expand the determinant:
\begin{align}
    \det(A - \lambda I) &= (2 - \lambda)(\lambda^2 - 7\lambda + 12)\notag \\
    &= 2\lambda^2 - 14\lambda + 24 - \lambda^3 + 7\lambda^2 - 12\lambda\notag \\
    &= -\lambda^3 + 9\lambda^2 - 26\lambda + 24\notag
\end{align}
Therefore, the characteristic equation is:
\begin{align}
    \lambda^3 - 9\lambda^2 + 26\lambda - 24 =  0
\end{align}
The roots of this characteristic equation are the eigenvalues. Simultaneously solving this polynomial equation, we obtain the roots, i.e., the distinct eigenvalues:
\begin{align}
    \lambda_1 = 2, \quad \lambda_2 = 3, \quad \lambda_3 = 4
\end{align}
For each eigenvalue $\lambda_i$, solve the equation:
\begin{align}
    (A - \lambda_i I) \mathbf{v} = 0,
\end{align}
where $\mathbf{v}$ is the eigenvector corresponding to $\lambda_i$.

Eigenvector for $\lambda_1 = 2$

Substitute $\lambda_1 = 2$:
\begin{align}
    A - 2I = \begin{bmatrix}
    2 - 2 & 1 & 0 \\
    0 & 3 - 2 & 0 \\
    1 & 0 & 4 - 2
    \end{bmatrix} = \begin{bmatrix}
    0 & 1 & 0 \\
    0 & 1 & 0 \\
    1 & 0 & 2
    \end{bmatrix}\notag
\end{align}
The equation $(A - 2I) \mathbf{v} = 0$ becomes:
\begin{align}
    \begin{bmatrix}
    0 & 1 & 0 \\
    0 & 1 & 0 \\
    1 & 0 & 2
    \end{bmatrix} \begin{bmatrix}
    v_1 \\
    v_2 \\
    v_3
    \end{bmatrix} = \begin{bmatrix}
    v_2 \\
    v_2 \\
    v_1 + 2v_3
    \end{bmatrix} = \begin{bmatrix}
    0 \\
    0 \\
    0
    \end{bmatrix}\notag
\end{align}
This gives $v_2 = 0$, and $v_1 + 2v_3 = 0 \Rightarrow v_1 = -2v_3$. Thus, the eigenvector is:
\begin{align}
    \mathbf{v}_1 = v_3 \begin{bmatrix}
    -2 \\
    0 \\
    1
    \end{bmatrix}\notag
\end{align}
Normalize $\mathbf{v}_1$:
\begin{align}
    \|\mathbf{v}_1\| = \sqrt{(-2)^2 + 0^2 + 1^2} = \sqrt{5}\notag
\end{align}
So the normalized eigenvector is:
\begin{align}
    \mathbf{v}_1 = \frac{1}{\sqrt{5}} \begin{bmatrix}
    -2 \\
    0 \\
    1
    \end{bmatrix}
\end{align}
Eigenvector for $\lambda_2 = 3$

Substitute $\lambda_2 = 3$:

\begin{align}
    A - 3I = \begin{bmatrix}
    2 - 3 & 1 & 0 \\
    0 & 3 - 3 & 0 \\
    1 & 0 & 4 - 3
    \end{bmatrix} = \begin{bmatrix}
    -1 & 1 & 0 \\
    0 & 0 & 0 \\
    1 & 0 & 1
    \end{bmatrix}\notag
\end{align}
The equation $(A - 3I) \mathbf{v} = 0$ becomes:
\begin{align}
    \begin{bmatrix}
    -1 & 1 & 0 \\
    0 & 0 & 0 \\
    1 & 0 & 1
    \end{bmatrix} \begin{bmatrix}
    v_1 \\
    v_2 \\
    v_3
    \end{bmatrix} = \begin{bmatrix}
    -v_1 + v_2 \\
    0 \\
    v_1 + v_3
    \end{bmatrix} = \begin{bmatrix}
    0 \\
    0 \\
    0
    \end{bmatrix}\notag
\end{align}
This gives $v_1 = v_2$, and $v_1 = -v_3$. Thus, the eigenvector is:
\begin{align}
    \mathbf{v}_2 = v_1 \begin{bmatrix}
    1 \\
    1 \\
    -1
    \end{bmatrix}\notag
\end{align}
Normalize $\mathbf{v}_2$:
\begin{align}
    |\mathbf{v}_2| = \sqrt{1^2 + 1^2 + (-1)^2} = \sqrt{3}\notag
\end{align}
So the normalized eigenvector is:
\begin{align}
    \mathbf{v}_2 = \frac{1}{\sqrt{3}} \begin{bmatrix}
    1 \\
    1 \\
    -1
    \end{bmatrix}
\end{align}
Eigenvector for $\lambda_3 = 4$

Substitute $\lambda_3 = 4$:
\begin{align}
    A - 4I = \begin{bmatrix}
    2 - 4 & 1 & 0 \\
    0 & 3 - 4 & 0 \\
    1 & 0 & 4 - 4
    \end{bmatrix} = \begin{bmatrix}
    -2 & 1 & 0 \\
    0 & -1 & 0 \\
    1 & 0 & 0
    \end{bmatrix}\notag
\end{align}
The equation $(A - 4I) \mathbf{v} = 0$ becomes:
\begin{align}
    \begin{bmatrix}
    -2 & 1 & 0 \\
    0 & -1 & 0 \\
    1 & 0 & 0
    \end{bmatrix} \begin{bmatrix}
    v_1 \\
    v_2 \\
    v_3
    \end{bmatrix} = \begin{bmatrix}
    -2v_1 + v_2 \\
    -v_2 \\
    v_1
    \end{bmatrix} = \begin{bmatrix}
    0 \\
    0 \\
    0
    \end{bmatrix}\notag
\end{align}
This gives $v_2 = 0$, and $v_1 = 0$. So the eigenvector is:
\begin{align}
    \mathbf{v}_3 = v_3 \begin{bmatrix}
    0 \\
    0 \\
    1
    \end{bmatrix}\notag
\end{align}
Normalize $\mathbf{v}_3$:
\begin{align}
    |\mathbf{v}_3| = 1\notag
\end{align}
So the normalized eigenvector is:
\begin{align}
    \mathbf{v}_3 = \begin{bmatrix}
    0 \\
    0 \\
    1
    \end{bmatrix}
\end{align}
\begin{itemize}
    \item The characteristic equation is:
    \begin{align*}
        \lambda^3 - 9\lambda^2 + 26\lambda - 24 = 0\qquad\text{(\bf Answer)}
    \end{align*}
    \item The eigenvalues are:
    \begin{align*}
        \lambda_1 = 2, \quad \lambda_2 = 3, \quad \lambda_3 = 4\qquad\text{(\bf Answer)}
    \end{align*}
    \item The corresponding orthonormal eigenvectors are:
    \begin{align*}
        \mathbf{v}_1 &= \frac{1}{\sqrt{5}} \begin{bmatrix}
        -2 \\
        0 \\
        1
    \end{bmatrix}, \quad
    \mathbf{v}_2 = \frac{1}{\sqrt{3}} \begin{bmatrix}
    1 \\
    1 \\
    -1
    \end{bmatrix}, \quad
    \mathbf{v}_3 = \begin{bmatrix}
    0 \\
    0 \\
    1
    \end{bmatrix}\qquad\text{(\bf Answer)}
    \end{align*}
\end{itemize}