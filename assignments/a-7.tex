\allowdisplaybreaks
\begin{center}
	\hrule
	\vspace{.4cm}
	\Large\scshape\textbf{AQM 831 --- Advanced Quantum Mechanics-1}
\end{center}
{\textbf{Name:}\ \textsc{Akiful Islam Zawad} \hspace{\hfill} \textbf{Due Date:} September 4, 2024\\[5pt]
{ \textbf{Track Follower:}} \ 2 \hspace{\hfill} \textbf{Week Number:} 7 \\
	\hrule}
 %----------------------------
\paragraph*{Problem Set 7} %\hfill \newline
\\
Prove that the following limit exists:
\begin{align*}
    \lim_{m \to p} \frac{\cos(m\pi) J_m - J_{-m}}{\sin m\pi}, \quad p \in \mathbb{Z}.
\end{align*}
\bigskip\bigskip\hline\hline\bigskip
\section*{Solution}
In the given expression:
\begin{align*}
    \frac{\cos(m\pi) J_m - J_{-m}}{\sin m\pi}.
\end{align*}
$\displaystyle J_m$ and $\displaystyle J_{-m}$ are Bessel functions of the first kind, and $\displaystyle m$ and $\displaystyle p$ denotes the orders of the equations.

Bessel functions of the first kind $\displaystyle J_m(x)$ satisfy the following symmetry property:
\begin{align}
    J_{-m}(x) &= (-1)^m J_m(x).\label{eq:bessel-symmetry-property}
\end{align}
Using (\ref{eq:bessel-symmetry-property}), we can rewrite $\displaystyle J_{-m}$ in terms of $\displaystyle J_m$:
\begin{align*}
    J_{-m} &= (-1)^m J_m.
\end{align*}
Substituting the above expression into the limit, we get:
\begin{align*}
    \lim_{m \to p} \frac{\cos(m\pi) J_m - (-1)^m J_m}{\sin m\pi}&=\lim_{m \to p} \frac{J_m \left(\cos(m\pi) - (-1)^m\right)}{\sin m\pi}.
\end{align*}
Now, observe that, so:
\begin{align*}
    \cos(m\pi) - (-1)^m &= (-1)^m - (-1)^m = 0.
\end{align*}
This implies that the numerator is 0, and hence, the limit reduces to:
\begin{align*}
    \lim_{m \to p} \frac{0}{\sin m\pi}=0.
\end{align*}
Thus, the limit exists and is equal to zero.
\bigskip\bigskip\hline\hline\bigskip
\begin{enumerate}
    \item Prove that:
    \begin{align*}
        J_{n-1} + J_{n+1} = \frac{2n}{x} J_n.
    \end{align*}
\bigskip\bigskip\hline\hline\bigskip
\section*{Solution}
We utilize the Bessel function of the first kind to solve the identities in this question. But first, we must find the two recurrence relation the Bessel function satisfy.

First, start with the Bessel function of the first kind of order $n$:
\begin{align}
    J_n=\sum_{r=0}^\infty\frac{(-1)^r}{r!\Gamma(n+r+1)}\left(\frac{x}{2}\right)^{n+2r}\label{eq:bessel-function-first-order}
\end{align}
Differentiating with respect to $x$, we get
\begin{align}
    J^{\prime}_n&=\dfrac{d}{dx}\left\{\sum_{r=0}^\infty\frac{(-1)^r}{r!\Gamma(n+r+1)}\left(\frac{x}{2}\right)^{n+2r}\right\}\notag\\
    &=n\sum_{r=0}^\infty\frac{(-1)^r(n+2r)}{r!\Gamma(n+r+1)}\left(\frac{x}{2}\right)^{n+2r-1}\cdot\frac{1}{2}\notag\\
    xJ^{\prime}_n&=n\sum_{r=0}^\infty\frac{(-1)^r}{r!\Gamma(n+r+1)}\left(\frac{x}{2}\right)^{n+2r} + x\sum_{r=0}^\infty\frac{(-1)^r\cdot2r}{2r!\Gamma(n+r+1)}\left(\frac{x}{2}\right)^{n+2r-1}\notag\\
    &=nJ_n + x\sum_{r=0}^\infty\frac{(-1)^r}{(r-1)!\Gamma(n+r+1)}\left(\frac{x}{2}\right)^{n+2r-1}\notag\\
    &=nJ_n+x\sum_{s=0}^\infty\frac{(-1)^{s+1}}{s!\Gamma(n+s+2)}\left(\frac{x}{2}\right)^{n+2s-1}\qquad[\text{Use  }r-1=s]\notag\\
    &=nJ_n-x\sum_{s=0}^\infty\frac{(-1)^s}{s!\Gamma[(n+1)+(s+1)]}\left(\frac{x}{2}\right)^{(n+1)+2s}\notag\\
    xJ^{\prime}_n&=nJ_n-xJ_{n+1}\label{eq:first-recurrence}
\end{align}
We follow a similar route to find the second recurrence relation:
\begin{align}
    J^{\prime}_n&=\dfrac{d}{dx}\left\{\sum_{r=0}^\infty\frac{(-1)^r}{r!\Gamma(n+r+1)}\left(\frac{x}{2}\right)^{n+2r}\right\}\notag\\
    &=\sum_{r=0}^\infty\frac{(-1)^r(n+2r)}{r!\Gamma(n+r+1)}\left(\frac{x}{2}\right)^{n+2r-1}\frac{1}{2}\notag\\
    &=\sum_{r=0}^\infty\frac{(-1)^r(n+2r)}{r!\Gamma(n+r+1)}\left(\frac{x}{2}\right)^{n+2r}\notag\notag\\
    &=\sum_{r=0}^\infty\frac{(-1)^r[(2n+2r)-n]}{r!\Gamma(n+r+1)}\left(\frac{x}{2}\right)^{n+2r}\notag\\
    &=\sum_{r=0}^\infty\frac{(-1)^r(2n+2r)}{r!\Gamma(n+r+1)}\left(\frac{x}{2}\right)^{n+2r}-n\sum_{r=0}^\infty\frac{(-1)^r}{r!\Gamma(n+r+1)}\left(\frac{x}{2}\right)^{n+2r}\notag\\
    &=\sum_{r=0}^\infty\frac{(-1)^r\cdot2}{r!\Gamma(n+r+1)}\left(\frac{x}{2}\right)^{n+2r}-nJ_n\notag\\
    &=x\sum_{r=0}^\infty\frac{(-1)^r}{r!\Gamma[(s-1)+r+1]}\left(\frac{x}{2}\right)^{(s-1)+2r}-nJ_n\qquad[\text{Use  }n=s-1].\notag\\
    xJ^{\prime}_n&=xJ_{n-1}-nJ_n.\label{eq:second-reccurence}
\end{align}
Equations (\ref{eq:first-recurrence}) and (\ref{eq:second-reccurence}) are the desired recurrence relations (\ref{eq:bessel-function-first-order}) satisfy.

Subtract (\ref{eq:second-reccurence}) from (\ref{eq:first-recurrence}) to obtain:
\begin{align}
    0& = 2nJ_n - xJ_{n+1} - xJ_{n-1}\notag\\
    2nJ_n&=x\left(J_{n-1}+J_{n+1}\right)\notag\\
    \therefore J_{n-1} + J_{n+1} &= \frac{2n}{x} J_n\qquad\text{(\bf Proved)}
\end{align}
\bigskip\bigskip\hline\hline\bigskip
\item Prove that:
\begin{align*}
    J_n' = \frac{1}{2} \left(J_{n-1} - J_{n+1}\right),
\end{align*}
\bigskip\bigskip\hline\hline\bigskip
\section*{Solution}
Subtract (\ref{eq:first-recurrence}) from (\ref{eq:second-reccurence}) to obtain:
\begin{align}
    2xJ^{\prime}_n & = - xJ_{n+1} + xJ_{n-1}\notag\\
    \therefore J_n' &= \frac{1}{2} \left(J_{n-1} - J_{n+1}\right)\qquad\text{(\bf Proved)}
\end{align}
\bigskip\bigskip\hline\hline\bigskip
\item Prove that:
\begin{align*}
    \frac{d}{dx}\left( x J_n J_{n+1} \right) = x \left( J_n^2 - J_{n+1}^2 \right).
\end{align*}
\bigskip\bigskip\hline\hline\bigskip
\section*{Solution}
We will apply the product rule of differentiation, which states that for any functions $\displaystyle u(x)$ and $\displaystyle v(x)$,
\begin{align}
    \frac{d}{dx} \left( u(x) v(x) \right) = u'(x) v(x) + u(x) v'(x).
\end{align}
Here, $\displaystyle u(x) = x$ and $\displaystyle v(x) = J_n(x) J_{n+1}(x)$.

Using the product rule:
\begin{align}
    \frac{d}{dx}\left( x J_n(x) J_{n+1}(x) \right) &= \frac{d}{dx}(x) \cdot J_n(x) J_{n+1}(x) + x \cdot \frac{d}{dx}\bigg[J_n(x) J_{n+1}(x)\bigg]\notag\\
    &= J_n(x) J_{n+1}(x) + x \cdot \frac{d}{dx}\bigg[J_n(x) J_{n+1}(x)\bigg]\notag\\[3pt]
    &= J_n(x) J_{n+1}(x) + x\bigg[J_n'(x) J_{n+1}(x) + J_n(x) J_{n+1}'(x)\bigg]\notag\\
    &= J_n(x) J_{n+1}(x) + xJ_n'(x) J_{n+1}(x) + J_n(x)x J_{n+1}'(x)\notag\\
    &= J_n(x) J_{n+1}(x) + J_{n+1}(x)\underbrace{\bigg[nJ_n(x)-xJ_{n+1}(x)\bigg]}_{(\ref{eq:first-recurrence})} + J_n(x)\underbrace{\bigg[-(n+1)J_{n+1}(x) + xJ_n(x)\bigg]}_{(\ref{eq:second-reccurence})\text{ with }n^{\prime} =\,n+1} \notag\\
    &= J_n(x) J_{n+1}(x) + nJ_n(x)J_{n+1}(x) - xJ_{n+1}^2 - (n+1)J_n(x)J_{n+1}(x) + xJ_n^2(x)\notag\\[5pt]
    &= J_n(x) J_{n+1}(x) + nJ_n(x)J_{n+1}(x) - xJ_{n+1}^2 - nJ_n(x)J_{n+1}(x) - J_n(x)J_{n+1}(x) + xJ_n^2(x)\notag\\[5pt]
    &= x \bigg[J_n^2(x) - J_{n+1}^2(x) \bigg].\qquad\text{(\bf Proved)}
\end{align}
\bigskip\bigskip\hline\hline\bigskip
\item Prove that:
\begin{align*}
    J_{\frac{1}{2}}(x) = \sqrt{\frac{2}{\pi x}} \sin x,
\end{align*}
\bigskip\bigskip\hline\hline\bigskip
\section*{Solution}
The Bessel function of the first kind of order $\displaystyle \frac{1}{2}$ is given by the series representation (\ref{eq:bessel-function-first-order}):
\begin{align}
    J_{\frac{1}{2}}&=\sum_{r=0}^\infty\frac{(-1)^r}{\displaystyle r!\Gamma\left(\frac{1}{2}+r+1\right)}\left(\frac{x}{2}\right)^{\frac{1}{2}+2r}\notag\\
    \left(\frac{x}{2}\right)^\frac{1}{2}J_{\frac{1}{2}}&=\sum_{r=0}^\infty\frac{(-1)^r}{\displaystyle r!\Gamma\left(\frac{3}{2}+r\right)}\left(\frac{x}{2}\right)^{2r+1}
\end{align}
We use the property of the Gamma function for half-integer values:
\begin{align}
    \Gamma\left(\frac{3}{2} + r\right) &= \Gamma\left(r + \frac{3}{2}\right) = \left(r+\dfrac12\right)\left(r-\dfrac12\right) \cdots \dfrac12 \sqrt{\pi} \notag\\
    &= \dfrac{(2r+1)(2r-1)\cdots 3 \cdot 1}{2^{r+1}} \sqrt{\pi}\notag\\
    &= \dfrac{(2r+1)!}{2^{2r+1}r!} \sqrt{\pi}.
\end{align}
Substituting this into the series:
\begin{align}
    J_{\frac{1}{2}}(x)\left(\frac{x}{2}\right)^{\frac{1}{2}} &= \frac{1}{\sqrt{\pi}}\sum_{k=0}^{\infty} \frac{(-1)^r 2^{2r+1}r!}{r!(2r + 1)!} \left(\frac{x}{2}\right)^{2r+1}\notag\\
    &=\frac{r!2^{2r+1}}{\sqrt{\pi}r!}\left(\sum_{r=0}^{\infty} \frac{(-1)^r}{(2r+1)!} x^{2r+1}\right)\left(\frac{1}{2}\right)^{{2r+1}}\notag\\
    &=\left(\frac{2}{x}\right)^{\frac{1}{2}}\frac{1}{\sqrt{\pi}r!}\left(\sum_{r=0}^{\infty} \frac{(-1)^r}{(2r+1)!} x^{2r+1}\right)\notag\\
    &=\sqrt{\frac{2}{\pi x}} \sin x.\qquad\text{(\bf Proved)}
\end{align}
We used the series representation of $\displaystyle \sin x$ given by:
\begin{align*}
    \sin x &= \sum_{k=0}^{\infty} \frac{(-1)^k}{(2k+1)!} x^{2k+1}.
\end{align*}
\end{enumerate}